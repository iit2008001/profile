\documentclass[a4paper,11pt]{article}
\usepackage[english]{babel}
\usepackage{graphicx, ctable, url}
\usepackage{longtable}
\usepackage{hyperref}
\usepackage{lscape}
\usepackage{subfig}
\usepackage{afterpage}
\usepackage{tabu}
\usepackage{listings} % source listings
\lstloadlanguages{java}
\lstset{keywordstyle=\ttfamily\bfseries}
\lstset{flexiblecolumns=true}
\lstset{commentstyle=\ttfamily\itshape}

\newcounter{magicrownumbers}
\newcommand \nextnumber{\stepcounter{magicrownumbers}\arabic{magicrownumbers}} %for counter for  agendas.  the variable \nextnumber increments automatically.

\begin{document}

\title{Minutes Meeting}
\author{Author}
\date{\today}

\maketitle	
\section{Meeting Information}
			\textbf{Date:} \\
			\textbf{Time:} \\
			\textbf{Place:}\\
	\subsection{Members Present}
Following members were present:
			\begin{itemize}
				\item Author 
				\item
			\end{itemize}
		\subsection{Member Apologies}
Following members were absent:
				\begin{itemize}
				\item 
				\end{itemize}
		
	\subsection{Revision History}
			\begin{tabular}{c | l | l }
				\textbf{Rev.} & \textbf{Date} & \textbf{Description} \\
				\hline
				1.0 &  & First draft\\
				1.1 & & Second draft \\
				

			\end{tabular}		

%Agenda
%Discussion
%Action
\newpage
\section{Discussion}
\setcounter{magicrownumbers}{0} % use this whenever you want to reset a counter. argument to the function is the value you want to initialise it. 
\begin{itemize}
	\item [\textbf{Agenda \nextnumber}]  
	\item [Discussion]
	\item [Action]
\end{itemize}

\begin{itemize}
	\item [\textbf{Agenda \nextnumber }] 
	\item [Discussion]  	
	
	\item [Action] 
\end{itemize}



  
\section{Next Meeting}
		\textbf{Date:} To be decided\\
		\textbf{Time:} To be decided \\
		\textbf{Place:} To be decided\\
	
	\subsection{Agenda}
				
\begin{thebibliography}{99}
\bibitem{site1} 
\bibitem{site2} 
\end{thebibliography}	
		
\end{document}
